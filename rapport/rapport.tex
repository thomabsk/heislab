\documentclass{article}
\usepackage[utf8]{inputenc}
\usepackage{amsmath}
\usepackage{graphicx}
\usepackage[parfill]{parskip}
\usepackage{fancyhdr}

\usepackage{biblatex}
\addbibresource{ref.bib}


\renewcommand{\figurename}{Figur}
\renewcommand{\tablename}{Tabell}


\begin{document}
\pagenumbering{gobble}
\newcommand{\HRule}{\rule{\linewidth}{0.5mm}}

\begin{center}

\textsc{\LARGE NTNU}\\[1.5cm] 
\textsc{\Large TTK4235}\\[0.5cm]
\textsc{\large Labgruppe 30}\\[0.5cm]

\HRule \\[0.4cm]
{ \huge \bfseries Rapport heislab}\\[0.4cm]
\HRule \\[1.5cm]
 
\begin{minipage}{0.6\textwidth}
\begin{flushleft} \large
\emph{Authors:}\\
Thomas \textsc{Borge Skøien} \\Vegard \textsc{Haraldstad} \\
\end{flushleft}
\end{minipage}
~
\\[2cm]

{\large \today}\\[2cm] 

\includegraphics[width=50mm]{img/ntnu_logo.png}

\end{center}
\newpage
\cfoot{\normalsize\thepage}
\pagenumbering{roman}

\section*{Introduksjon}

\tableofcontents

\newpage
\nocite{*}
\cfoot{\normalsize\thepage~}
\pagenumbering{arabic}
\section{Overordnet arkitektur}
\section{Moduldesign}
\subsection{Control}
\subsection{Queue}  
Kømodulen vår ble designet med mål om å være enkel. Selve køen er implementert som tre arrays av ints, en for bestillinger opp, en for ned og en for bestillinger fra heisvognen, hvor hvert element i et av arrayene representerer hvovidt vi har en type bestilling, og i hvilken etasje, eller ikke. 

\subsection{Elevator}
\subsection{Timer}
\section{Testing}
\section{Diskusjon}
\end{document}